\documentclass{article}

\usepackage[a4paper,margin=1in]{geometry}
\usepackage{amsmath,amssymb}
\usepackage{booktabs}

\title{\textbf{CSE400 – Fundamentals of Probability in Computing}}
\author{\textbf{Kensi Patel - AU2440102}}
\date{Lecture 11}

\begin{document}

\maketitle

\textbf{Lecture 11: Transformation of Random Variables}

Instructor: Dhaval Patel, PhD

\section*{1. Problem Setting and Notation}
\begin{itemize}
    \item Given a continuous random variable $X$ with known PDF $f_X(x)$.
    \item Define a new random variable:\[Y = g(X)\]
    \item Goal: Find $F_Y(y)$ and $f_Y(y)$.
\end{itemize}


\section*{2. Lecture Flow and Logical Dependencies}
\begin{enumerate}
    \item Transformation of one random variable
    \item Function of two random variables
    \item Example: $Z=X+Y$
\end{enumerate}
Logical progression:
\[\text{Single RV} \rightarrow \text{Joint RV} \rightarrow \text{Example}\]


\section*{3. Transformation of One Random Variable}

\subsection*{3.1 Assumptions}
\begin{itemize}
    \item Transformation: \[Y=g(X)\]
    \item Assume $g(\cdot)$ is monotonic (increasing or decreasing), so the inverse exists:\[X = g^{-1}(Y)\]
\end{itemize}

\subsection*{3.2 Step 1: CDF Method}
\begin{itemize}
    \item Definition:\[F_Y(y)=P(Y\le y)\]
    \item Substitute transformation:\[F_Y(y)=P(g(X)\le y)\]
    \item If $g$ is increasing:\[F_Y(y)=P\left(X\le g^{-1}(y)\right)\]
    \item Hence:\[F_Y(y)=F_X\big(g^{-1}(y)\big)\]
\end{itemize}

\subsection*{3.3 Step 2: PDF from CDF}
\begin{itemize}
    \item Differentiate:\[f_Y(y)=\frac{d}{dy}F_Y(y)\]
    \item Substitute:\[f_Y(y)=\frac{d}{dy}F_X\big(g^{-1}(y)\big)\]
    \item Using the chain rule:\[f_Y(y)=f_X\big(g^{-1}(y)\big)\cdot \frac{d}{dy}g^{-1}(y)\]
    \item Since\[\frac{d}{dy}g^{-1}(y)=\frac{dx}{dy},\]
    \item Final transformation formula:\[f_Y(y)=f_X(x)\left|\frac{dx}{dy}\right|_{x=g^{-1}(y)}\]

\end{itemize}

\subsection*{3.4 Decreasing Transformation Case}
\begin{itemize}
    \item If $g(\cdot)$ is decreasing:\[F_Y(y)=P(X\ge g^{-1}(y))=1-F_X\big(g^{-1}(y)\big)\]
    \item Differentiate:\[f_Y(y)=-f_X(x)\frac{dx}{dy}\]
    \item Taking magnitude gives a unified result:\[f_Y(y)=f_X(x)\left|\frac{dx}{dy}\right|\]
\end{itemize}


\section*{4. Worked Example}
\begin{itemize}
    \item Given:\[X\sim U(-1,1)\]
    \item PDF:\[f_X(x)=\begin{cases}\dfrac{1}{2}, & -1<x<1\\0, & \text{otherwise}\end{cases}\]
    \item Transformation:\[Y=\sin\left(\frac{\pi X}{2}\right)\]
\end{itemize}

\subsection*{4.1 Inverse Mapping}
\[x=\frac{2}{\pi}\sin^{-1}(y)\]

\subsection*{4.2 Differentiation}
\[\frac{dx}{dy}=\frac{2}{\pi}\frac{1}{\sqrt{1-y^2}}\]

\subsection*{4.3 PDF of $Y$}
\begin{itemize}
    \item Apply transformation rule:\[f_Y(y)=f_X(x)\left|\frac{dx}{dy}\right|\]
    \item Substitute $f_X(x)=\frac{1}{2}$:\[f_Y(y)=\frac{1}{2}\cdot \frac{2}{\pi}\frac{1}{\sqrt{1-y^2}}\]
    \item Final result:\[f_Y(y)=\frac{1}{\pi\sqrt{1-y^2}}, \quad -1<y<1\]
    \item Limits from mapping:\[x=-1 \Rightarrow y=-1, \qquad x=1 \Rightarrow y=1\]
\end{itemize}


\section*{5. Function of Two Random Variables}

\begin{itemize}
    \item Define:\[Z=X+Y\]
\end{itemize}

Goal: Find $f_Z(z)$.

\textbf{Examples mentioned:}
\begin{itemize}
    \item If $X,Y\sim N(0,1)$ then $Z\sim N(0,2)$
    \item Exponential RV case discussed
\end{itemize}


\section*{6. Distribution of $Z=X+Y$}

\subsection*{6.1 Start from CDF}
\[F_Z(z)=P(Z\le z)=P(X+Y\le z)\]

\subsection*{6.2 Convert to Double Integral}
\[F_Z(z)=\iint_{x+y\le z} f_{XY}(x,y)\,dx\,dy\]
Region defined by boundary:\[x+y=z\]

\subsection*{6.3 Change Order of Integration}

Equivalent expression:\[F_Z(z)=\int_{-\infty}^{\infty}\int_{-\infty}^{z-x}f_{XY}(x,y)\,dy\,dx\]


\section*{7. Logical Flow Summary}

\begin{itemize}
    \item Start from CDF definition
    \item Use inverse mapping
    \item Differentiate to obtain PDF
    \item Extend idea to joint RVs
    \item Apply to $Z=X+Y$
\end{itemize}


\section*{8. Final Key Formulas}
\subsection*{Single RV Transformation}
CDF Transformation
\[F_Y(y)=F_X\big(g^{-1}(y)\big)\]
PDF Transformation
\[f_Y(y)=f_X\big(g^{-1}(y)\big)\left|\frac{dx}{dy}\right|\]

\subsection*{Sum of Two Random Variables}
Sum CDF
\[F_Z(z)=\iint_{x+y\le z} f_{XY}(x,y)\,dx\,dy\]

\end{document}
