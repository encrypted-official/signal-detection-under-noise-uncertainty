\documentclass[12pt,a4paper]{article}

\usepackage[a4paper,margin=1in]{geometry}
\usepackage{amsmath,amssymb,amsthm}
\usepackage{fancyhdr}
\usepackage{graphicx}
\usepackage{enumitem}
\usepackage{setspace}

\setstretch{1.2}

\pagestyle{fancy}
\fancyhf{}
\fancyhead[L]{Name - Varun Hotani}
\fancyhead[R]{Roll No. - AU2440132}
\fancyfoot[C]{\thepage}

\title{\textbf{CSE400: Fundamentals of Probability in Computing}\\
\Large Lecture 11: Transformation of Random Variables}
\author{}
\date{}

\begin{document}

\maketitle

\section*{Lecture Information}

\begin{itemize}
    \item Course: CSE400 - Fundamentals of Probability in Computing
    \item Topic: Transformation of Random Variables
    \item Instructor: Dhaval Patel, PhD
    \item Date: February 10, 2026
\end{itemize}

\newpage

\tableofcontents
\newpage

%--------------------------------------------------
\section{Introduction}

In this lecture, we study:

\begin{enumerate}
    \item Transformation of Random Variables
    \item Function of Two Random Variables
    \item Illustrative Example: $Z = X + Y$
\end{enumerate}

The goal is to learn systematic techniques to determine the probability distribution of a new random variable obtained as a function of one or more given random variables.

%--------------------------------------------------
\section{Transformation of Random Variables}

\subsection{Basic Definition}

Let $X$ be a random variable with known probability distribution.

Let a new random variable $Y$ be defined as:
\[
Y = g(X)
\]

where $g(\cdot)$ is a function of $X$.

Our objective:
\[
\text{Find the distribution of } Y
\]

\subsection{Method for Continuous Random Variables}

Assume:
\begin{itemize}
    \item $X$ is continuous
    \item $g(\cdot)$ is differentiable and monotonic
\end{itemize}

We proceed step-by-step.

\subsubsection*{Step 1: CDF Approach}

Define:
\[
F_Y(y) = P(Y \le y)
\]

Since $Y = g(X)$,
\[
F_Y(y) = P(g(X) \le y)
\]

If $g$ is strictly increasing:

\[
P(g(X) \le y) = P(X \le g^{-1}(y))
\]

Thus,

\[
F_Y(y) = F_X(g^{-1}(y))
\]

\subsubsection*{Step 2: Differentiate to Get PDF}

\[
f_Y(y) = \frac{d}{dy}F_Y(y)
\]

Using chain rule:

\[
f_Y(y) = f_X(g^{-1}(y)) \left| \frac{d}{dy} g^{-1}(y) \right|
\]

\subsection{Final Formula (Single Variable Transformation)}

\[
\boxed{
f_Y(y) = f_X(x)\left| \frac{dx}{dy} \right|
}
\]

where $x = g^{-1}(y)$.

This formula is fundamental for exams.

%--------------------------------------------------
\section{Function of Two Random Variables}

Now consider two random variables:

\[
Z = h(X,Y)
\]

where $(X,Y)$ has joint distribution $f_{X,Y}(x,y)$.

Our goal:
\[
\text{Find the distribution of } Z
\]

\subsection{Joint Transformation Method}

Let:

\[
U = g_1(X,Y), \quad V = g_2(X,Y)
\]

We compute the joint density of $(U,V)$ using the Jacobian method.

\subsection{Jacobian Definition}

If the transformation is invertible:

\[
(x,y) \rightarrow (u,v)
\]

Then:

\[
f_{U,V}(u,v) = f_{X,Y}(x,y)\left|J\right|
\]

where Jacobian:

\[
J =
\begin{vmatrix}
\frac{\partial x}{\partial u} & \frac{\partial x}{\partial v} \\
\frac{\partial y}{\partial u} & \frac{\partial y}{\partial v}
\end{vmatrix}
\]

\subsection{Procedure}

\begin{enumerate}
    \item Define transformation
    \item Find inverse transformation
    \item Compute Jacobian determinant
    \item Substitute into joint density
    \item Integrate out unwanted variable if needed
\end{enumerate}

%--------------------------------------------------
\section{Illustrative Example: $Z = X + Y$}

We now derive distribution of:

\[
Z = X + Y
\]

\subsection{Assumptions}

Let:
\begin{itemize}
    \item $X$ and $Y$ be continuous random variables
    \item Joint density: $f_{X,Y}(x,y)$
\end{itemize}

\subsection{CDF Derivation}

Start from definition:

\[
F_Z(z) = P(Z \le z)
\]

Since $Z = X + Y$,

\[
F_Z(z) = P(X + Y \le z)
\]

\subsection{Region Interpretation}

The event:

\[
X + Y \le z
\]

represents a region in the $xy$-plane.

Thus,

\[
F_Z(z) =
\int\int_{x+y\le z} f_{X,Y}(x,y)\, dx\, dy
\]

\subsection{PDF of Z}

Differentiate:

\[
f_Z(z) = \frac{d}{dz}F_Z(z)
\]

By differentiation under the integral sign:

\[
f_Z(z) =
\int_{-\infty}^{\infty} f_{X,Y}(x,z-x)\, dx
\]

\subsection{If $X$ and $Y$ Are Independent}

If independent:

\[
f_{X,Y}(x,y) = f_X(x)f_Y(y)
\]

Substitute:

\[
f_Z(z) =
\int_{-\infty}^{\infty} f_X(x)f_Y(z-x)\, dx
\]

\subsection{Final Result (Convolution)}

\[
\boxed{
f_Z(z) = (f_X * f_Y)(z)
=
\int_{-\infty}^{\infty}
f_X(x)f_Y(z-x)\, dx
}
\]

This is called the \textbf{convolution} of two densities.

\newpage

%--------------------------------------------------
\section{Logical Understanding for Exams}

\subsection{Why Convolution Appears}

Because:

\begin{itemize}
    \item $Z=z$ occurs when $X=x$ and $Y=z-x$
    \item We integrate over all possible $x$
\end{itemize}

Thus probability accumulates across all decompositions of $z$.

\subsection{Decision Logic}

\begin{itemize}
    \item If $Y = g(X)$ → Use single-variable transformation formula.
    \item If $Z = X + Y$ and independent → Use convolution.
    \item If general transformation of two variables → Use Jacobian method.
\end{itemize}

%--------------------------------------------------
\section{Summary for Revision}

\subsection*{Key Formulas}

\textbf{1. Single Variable Transformation:}

\[
f_Y(y) = f_X(x)\left|\frac{dx}{dy}\right|
\]

\textbf{2. Joint Transformation (Jacobian):}

\[
f_{U,V}(u,v) = f_{X,Y}(x,y)|J|
\]

\textbf{3. Sum of Independent Variables:}

\[
f_Z(z) = \int f_X(x)f_Y(z-x)\,dx
\]

\subsection*{Exam Strategy}

\begin{enumerate}
    \item Identify transformation type.
    \item Check independence.
    \item Decide between:
    \begin{itemize}
        \item Inverse + derivative method
        \item Jacobian method
        \item Convolution method
    \end{itemize}
    \item Clearly define integration limits.
    \item Always justify steps.
\end{enumerate}

%--------------------------------------------------

\section*{End of Lecture 11 Scribe}

\end{document}
