\documentclass[12pt]{article}

\usepackage[a4paper,margin=1in]{geometry}
\usepackage{amsmath,amssymb,amsthm}
\usepackage{graphicx}
\usepackage{fancyhdr}
\usepackage{hyperref}

\setlength{\parskip}{6pt}
\setlength{\parindent}{0pt}

\pagestyle{fancy}
\fancyhf{}
\lhead{\textbf{Name- Parthiv Karangiya \quad Roll no.- AU2440016}}
\rhead{\textbf{CSE400 Lecture 11}}
\cfoot{\thepage}

\title{\textbf{CSE400: Fundamentals of Probability in Computing}\\
\Large Lecture 11: Transformation of Random Variables}
\author{\textbf{Dhaval Patel, PhD}\\Associate Professor\\SEAS, Ahmedabad University}
\date{February 10, 2026}

\begin{document}

\maketitle

\section{Lecture Outline}
This lecture covers the following major topics:
\begin{enumerate}
    \item Transformation of Random Variables
    \item Function of Two Random Variables
    \item Illustrative Example: Detailed derivation for the case $Z = X + Y$
\end{enumerate}

\section{Transformation of Random Variables}

\subsection{Basic Setup}
Assume we have a continuous random variable $X$ whose distribution is known.

We assume:
\[
f_X(x) \quad \text{(PDF of $X$ is known)}
\]
and
\[
F_X(x) \quad \text{(CDF of $X$ is known)}
\]

Now define a new random variable $Y$ as a function of $X$:
\[
Y = g(X)
\]

The objective is to find:
\[
F_Y(y) \quad \text{and} \quad f_Y(y)
\]

\subsection{Important Idea: Inverting the Transformation}
If $Y = g(X)$ is invertible, then we can write:
\[
X = g^{-1}(Y)
\]

This inversion is the key step in computing the CDF and PDF of $Y$.

\section{CDF Method for Transformation}

\subsection{Step 1: Derive the CDF of $Y$}
The CDF of $Y$ is defined as:
\[
F_Y(y) = P(Y \le y)
\]

Since $Y = g(X)$:
\[
F_Y(y) = P(g(X) \le y)
\]

At this stage, the monotonic nature of $g(\cdot)$ matters.

\subsection{Case 1: $g(\cdot)$ is Monotonically Increasing (+ve Increasing)}

If $g(\cdot)$ is increasing, then:
\[
g(X) \le y \iff X \le g^{-1}(y)
\]

So:
\[
F_Y(y) = P(X \le g^{-1}(y))
\]

Thus:
\[
F_Y(y) = F_X(g^{-1}(y))
\]

\subsection{Step 2: Differentiate to Get PDF}
Now:
\[
f_Y(y) = \frac{d}{dy}F_Y(y)
\]

Using chain rule:
\[
f_Y(y) = \frac{d}{dy}\Big(F_X(g^{-1}(y))\Big)
\]

So:
\[
f_Y(y) = f_X(g^{-1}(y)) \cdot \frac{d}{dy}\big(g^{-1}(y)\big)
\]

Let:
\[
x = g^{-1}(y)
\]

Then:
\[
f_Y(y) = f_X(x)\cdot \frac{dx}{dy}
\]

Since PDF must be non-negative, we write the final standard form:
\[
f_Y(y) = f_X(g^{-1}(y))\left|\frac{d}{dy}g^{-1}(y)\right|
\]

\subsection{Final PDF Transformation Formula (Monotone Case)}
\[
\boxed{
f_Y(y) = f_X(g^{-1}(y))\left|\frac{d}{dy}g^{-1}(y)\right|
}
\]

\section{Transformation when $g(\cdot)$ is Monotonically Decreasing (-ve Decreasing)}

\subsection{Step 1: CDF of $Y$}
Again:
\[
F_Y(y) = P(Y \le y) = P(g(X)\le y)
\]

If $g(\cdot)$ is decreasing, then:
\[
g(X)\le y \iff X \ge g^{-1}(y)
\]

So:
\[
F_Y(y) = P(X \ge g^{-1}(y))
\]

Using complement rule:
\[
P(X \ge a) = 1 - P(X < a)
\]

Thus:
\[
F_Y(y) = 1 - F_X(g^{-1}(y))
\]

\subsection{Step 2: Differentiate to Get PDF}
Differentiate:
\[
f_Y(y) = \frac{d}{dy}F_Y(y)
\]

So:
\[
f_Y(y) = \frac{d}{dy}\Big(1 - F_X(g^{-1}(y))\Big)
\]

\[
f_Y(y) = - f_X(g^{-1}(y))\cdot \frac{d}{dy}g^{-1}(y)
\]

Since the derivative is negative for decreasing transformations, we again take absolute value:

\[
\boxed{
f_Y(y) = f_X(g^{-1}(y))\left|\frac{d}{dy}g^{-1}(y)\right|
}
\]

\subsection{Conclusion}
Therefore, for both monotonic increasing and monotonic decreasing transformations, the same final PDF formula holds.

\section{Example: $X \sim U(-1,1)$ and $Y = \sin\left(\frac{\pi X}{2}\right)$}

\subsection{Given}
$X$ is uniformly distributed over $(-1,1)$:
\[
f_X(x) =
\begin{cases}
\frac{1}{2}, & -1 < x < 1\\
0, & \text{otherwise}
\end{cases}
\]

Transformation is:
\[
Y = \sin\left(\frac{\pi X}{2}\right)
\]

\subsection{Step 1: Invert the Transformation}
We solve for $x$:

\[
y = \sin\left(\frac{\pi x}{2}\right)
\]

Take inverse sine:
\[
\sin^{-1}(y) = \frac{\pi x}{2}
\]

Thus:
\[
x = \frac{2}{\pi}\sin^{-1}(y)
\]

So:
\[
g^{-1}(y) = \frac{2}{\pi}\sin^{-1}(y)
\]

\subsection{Step 2: Compute Derivative}
\[
\frac{dx}{dy} = \frac{2}{\pi}\cdot \frac{d}{dy}\sin^{-1}(y)
\]

We know:
\[
\frac{d}{dy}\sin^{-1}(y) = \frac{1}{\sqrt{1-y^2}}
\]

So:
\[
\frac{dx}{dy} = \frac{2}{\pi}\cdot \frac{1}{\sqrt{1-y^2}}
\]

\subsection{Step 3: Apply Transformation Formula}
\[
f_Y(y) = f_X(x)\left|\frac{dx}{dy}\right|
\]

Given $f_X(x)=\frac{1}{2}$ for $-1<x<1$.

Thus:
\[
f_Y(y) = \frac{1}{2}\cdot \left(\frac{2}{\pi}\cdot \frac{1}{\sqrt{1-y^2}}\right)
\]

\[
f_Y(y) = \frac{1}{\pi\sqrt{1-y^2}}
\]

\subsection{Final Answer}
\[
\boxed{
f_Y(y) = \frac{1}{\pi\sqrt{1-y^2}}, \quad -1<y<1
}
\]

\section{Function of Two Random Variables}

\subsection{General Setup}
Let $X$ and $Y$ be random variables.

We define a new random variable $Z$ as a function of two random variables:
\[
Z = g(X,Y)
\]

Common examples include:
\[
Z = X+Y,\quad Z = X-Y,\quad Z = \frac{X}{Y},\quad Z = XY
\]

The objective is to find the distribution of $Z$, typically:
\[
F_Z(z) \quad \text{or} \quad f_Z(z)
\]

\section{Illustrative Example: $Z = X + Y$}

\subsection{Goal}
Define:
\[
Z = X+Y
\]

We want to compute:
\[
F_Z(z) = P(Z \le z)
\]

That is:
\[
F_Z(z) = P(X+Y \le z)
\]

\subsection{CDF Derivation}
\[
F_Z(z) = P(X+Y \le z)
\]

This probability corresponds to a region in the $(x,y)$ plane:
\[
x+y \le z
\]

Equivalently:
\[
y \le z-x
\]

\subsection{Region-Based Integration Form}
To compute this probability using the joint PDF:
\[
f_{X,Y}(x,y)
\]

we integrate over the region:
\[
\{(x,y): x+y \le z\}
\]

Thus:
\[
F_Z(z) = \int\int_{x+y \le z} f_{X,Y}(x,y)\,dx\,dy
\]

\subsection{Integral Expression (As Written in Lecture)}
The lecture shows the CDF computed as:

\[
F_Z(z)=P(Z\le z)
\]

\[
F_Z(z)=P(X+Y\le z)
\]

and written as a double integral:

\[
F_Z(z) = \int_{-\infty}^{\infty}\int_{-\infty}^{z-x} f_{X,Y}(x,y)\,dy\,dx
\]

This represents:
\begin{itemize}
    \item outer integral over $x$
    \item inner integral over $y$ from $-\infty$ to $z-x$
\end{itemize}

\subsection{Alternative Order of Integration (As Shown)}
The lecture also illustrates changing the integration order using region geometry.

From the graph, the boundary line is:
\[
y = z-x
\]

So the integration can also be expressed by integrating first over $x$ with bounds depending on $y$.

The lecture representation indicates rewriting into a second integral form by changing limits, written as:

\[
F_Z(z) = \int_{y=-\infty}^{z}\int_{x=-\infty}^{z-y} f_{X,Y}(x,y)\,dx\,dy
\]

This matches the region:
\[
x \le z-y
\]

\subsection{Key Conclusion}
The derived CDF of $Z=X+Y$ is computed using region integration:

\[
\boxed{
F_Z(z)=\int\int_{x+y\le z} f_{X,Y}(x,y)\,dx\,dy
}
\]

and the corresponding limit form:

\[
\boxed{
F_Z(z) = \int_{-\infty}^{\infty}\int_{-\infty}^{z-x} f_{X,Y}(x,y)\,dy\,dx
}
\]

\section{Important Exam Notes from Lecture}
\begin{itemize}
    \item When transforming one random variable $Y=g(X)$, first find $F_Y(y)$ and then differentiate.
    \item The monotonic nature of $g(\cdot)$ determines whether inequality direction changes.
    \item Final PDF formula for monotone transformations is:
    \[
    f_Y(y)=f_X(g^{-1}(y))\left|\frac{d}{dy}g^{-1}(y)\right|
    \]
    \item For two random variables, distributions of expressions like $Z=X+Y$ are derived using joint probability region integration.
\end{itemize}

\section{Highlighted Problems Mentioned in Lecture (For Practice)}
The lecture lists the following tasks for $Z=X+Y$:
\begin{enumerate}
    \item Find PDF of $Z$, $f_Z(z)$.
    \item Find $f_Z(z)$ if $X$ and $Y$ are independent.
    \item If $X\sim N(0,1)$ and $Y\sim N(0,1)$ are independent, prove $Z\sim N(0,2)$.
    \item If $X$ and $Y$ are exponential RVs with parameter $\lambda$, find $f_Z(z)$.
\end{enumerate}

\textbf{Note:} The lecture slide only lists these as questions. No full derivation for these specific distributions is provided in the slides, hence they are not expanded here.

\section{End of Lecture 11 Notes}
This document strictly follows Lecture 11 slides and reconstructs the derivations as shown.

\end{document}
