\documentclass[11pt]{article}
\usepackage{amsmath,amssymb}
\usepackage{geometry}
\usepackage{setspace}
\usepackage{enumitem}
\geometry{margin=1in}
\setstretch{1.15}

\begin{document}

\begin{center}
{\Large \textbf{CSE 400 — Lecture 11 Scribe}}\\
\vspace{0.2em}
Fundamentals of Probability in Computing\\
Lecture 11: Transformation of Random Variables
\end{center}

\section{Transformation of Random Variables}

\subsection{Problem Setting}

Let $X$ be a continuous random variable (CRV) with known:

PDF: $f_X(x)$

CDF: $F_X(x)$

These are assumed known a priori.

Define a new random variable:
\[
Y = g(X)
\]

The objective is to determine:
\[
F_Y(y), \qquad f_Y(y)
\]

The transformation is treated under the assumption of monotonicity of $g(\cdot)$.

Two cases arise:
\begin{itemize}[noitemsep]
\item Monotonically increasing
\item Monotonically decreasing
\end{itemize}

\subsection{Step 1: CDF Method}

\[
F_Y(y) = P(Y \le y)
\]

Since $Y = g(X)$,
\[
F_Y(y) = P(g(X) \le y)
\]

The next step depends on monotonicity of $g(\cdot)$.

\subsubsection*{Case 1: Monotonically Increasing}

If $g$ is increasing:
\[
g(X) \le y \iff X \le g^{-1}(y)
\]

Thus,
\[
F_Y(y) = P(X \le g^{-1}(y)) = F_X(g^{-1}(y))
\]

Differentiate to obtain PDF:
\[
f_Y(y) = \frac{d}{dy} F_Y(y)
= \frac{d}{dy} F_X(g^{-1}(y))
\]

Using chain rule:
\[
f_Y(y) = f_X(g^{-1}(y)) \cdot \frac{d}{dy} g^{-1}(y)
\]

Equivalently written:
\[
f_Y(y) = f_X(x) \frac{dx}{dy} \quad \text{evaluated at } x = g^{-1}(y)
\]

\subsubsection*{Case 2: Monotonically Decreasing}

If $g$ is decreasing:
\[
g(X) \le y \iff X \ge g^{-1}(y)
\]

Thus,
\[
F_Y(y) = P(X \ge g^{-1}(y))
= 1 - F_X(g^{-1}(y))
\]

Differentiate:
\[
f_Y(y) = \frac{d}{dy} \left[1 - F_X(g^{-1}(y))\right]
\]

\[
= - f_X(g^{-1}(y)) \cdot \frac{d}{dy} g^{-1}(y)
\]

Since $\frac{d}{dy} g^{-1}(y) < 0$,
\[
f_Y(y) = f_X(x) \frac{dx}{dy} \quad \text{evaluated at } x = g^{-1}(y)
\]

\subsection{Change of Limits}

Adjust limits according to the mapping $y = g(x)$.

The support of $Y$ is determined by transforming the support of $X$.

\section{Worked Example}

Given:
\[
X \sim \text{Uniform}(-1,1)
\]

\[
f_X(x) =
\begin{cases}
\frac{1}{2}, & -1 < x < 1 \\
0, & \text{otherwise}
\end{cases}
\]

Define:
\[
Y = \sin\left(\frac{\pi X}{2}\right)
\]

Find $f_Y(y)$.

\subsection*{Step 1: Inverse Transformation}

\[
X = \frac{2}{\pi} \sin^{-1}(Y)
\]

\subsection*{Step 2: Derivative}

\[
\frac{dx}{dy} = \frac{2}{\pi} \cdot \frac{1}{\sqrt{1 - y^2}}
\]

\subsection*{Step 3: Apply Transformation Formula}

\[
f_Y(y) = f_X(x) \frac{dx}{dy}
\]

Substitute $f_X(x) = \frac{1}{2}$:

\[
f_Y(y) = \frac{1}{2} \cdot \frac{2}{\pi} \cdot \frac{1}{\sqrt{1-y^2}}
\]

\[
f_Y(y) = \frac{1}{\pi \sqrt{1-y^2}}
\]

Support:

Since $x \in (-1,1)$,
\[
y \in (-1,1)
\]

Final answer:
\[
f_Y(y) =
\begin{cases}
\dfrac{1}{\pi \sqrt{1-y^2}}, & -1 < y < 1 \\
0, & \text{otherwise}
\end{cases}
\]

\section{Function of Two Random Variables}

Let:
\[
Z = X + Y
\]

Goal:
\begin{enumerate}[label=\arabic*.]
\item Find $f_Z(z)$.
\item If $X$ and $Y$ are independent, find $f_Z(z)$.
\item If $X,Y \sim N(0,1)$, prove $Z \sim N(0,2)$.
\item If exponential with parameter $\lambda$, find $f_Z(z)$.
\end{enumerate}

\section{Derivation of $f_Z(z)$}

\subsection*{Step 1: CDF Method}

\[
F_Z(z) = P(Z \le z)
= P(X+Y \le z)
\]

Region:
\[
\{(x,y) : x + y \le z\}
\]

Line:
\[
x + y = z
\]

Region below the line.

\subsection*{Step 2: Double Integral Form}

\[
F_Z(z) = \int_{-\infty}^{\infty}
\int_{-\infty}^{z-y}
f_{XY}(x,y)\, dx\, dy
\]

Equivalently (order switched):
\[
F_Z(z) = \int_{-\infty}^{\infty}
\int_{-\infty}^{z-x}
f_{XY}(x,y)\, dy\, dx
\]

\subsection*{Step 3: If $X$ and $Y$ Independent}

\[
f_{XY}(x,y) = f_X(x) f_Y(y)
\]

Thus,
\[
f_Z(z) = \int_{-\infty}^{\infty}
f_X(x) f_Y(z-x)\, dx
\]

This is the convolution form for the PDF of $Z = X + Y$.

\section{Summary of Core Results}

\subsection*{Single Variable Transformation}

If $Y = g(X)$ and $g$ is monotone:
\[
f_Y(y) = f_X(g^{-1}(y)) \frac{d}{dy} g^{-1}(y)
\]

\subsection*{Sum of Two Random Variables}

\[
F_Z(z) = P(X+Y \le z)
\]

If independent:
\[
f_Z(z) = \int_{-\infty}^{\infty}
f_X(x) f_Y(z-x)\, dx
\]

\end{document}