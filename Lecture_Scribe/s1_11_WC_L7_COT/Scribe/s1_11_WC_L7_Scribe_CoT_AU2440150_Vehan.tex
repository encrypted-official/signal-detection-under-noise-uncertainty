\documentclass[11pt]{article}
\usepackage{amsmath,amssymb,amsthm}
\usepackage{geometry}
\usepackage{setspace}
\usepackage{enumitem}
\geometry{margin=1in}
\setstretch{1.15}

\begin{document}

\begin{center}
{\Large \textbf{CSE 400 — Lecture 7 Scribe}}\\
\vspace{0.2em}
Fundamentals of Probability in Computing\\
January 27, 2025
\end{center}

\section{Cumulative Distribution Function (CDF)}

\subsection{Intuition: Water Tank Analogy}

A water tank analogy is used to build intuition for the cumulative distribution function.

Let the height of water in the tank be denoted by $h$.

The volume of water up to height $h$ is denoted by $V(h)$.

The height $h$ is analogous to the value of a random variable.

The volume $V(h)$ is analogous to the cumulative probability up to that value.

Assume:
\begin{itemize}[noitemsep]
\item The tank has constant radius $R$.
\item Cross-sectional area is $\pi R^2$.
\end{itemize}

The volume up to height $h$ is:
\[
V(h) = \int_{0}^{h} \pi R^2 \, dh = \pi R^2 h
\]

Key analogies stated explicitly in the lecture:
\begin{itemize}[noitemsep]
\item $\pi R^2$ is analogous to a probability density function (PDF) of a uniform distribution.
\item The maximum volume of the tank is
\[
V(H) = \pi R^2 H
\]
which is analogous to total probability $=1$.
\end{itemize}

Thus:
\begin{itemize}[noitemsep]
\item CDF corresponds to accumulated volume.
\item PDF corresponds to rate of accumulation.
\end{itemize}

\subsection{Definition of the CDF}

\textbf{Definition.}
Let $X$ be a random variable. The cumulative distribution function (CDF) of $X$ is defined as
\[
F_X(x) = \Pr(X \le x), \quad -\infty < x < \infty
\]

The lecture explicitly states that most information about the random experiment described by $X$ is determined by the behavior of $F_X(x)$.

\subsection{Properties of the CDF}

\begin{enumerate}[label=\arabic*.]
\item \textbf{Bounds}
\[
0 \le F_X(x) \le 1
\]

\item \textbf{Limits at Infinity}
\[
F_X(-\infty) = 0, \qquad F_X(\infty) = 1
\]

\item \textbf{Monotonicity}

For $x_1 < x_2$,
\[
F_X(x_1) \le F_X(x_2)
\]

\item \textbf{Interval Probability}

For $x_1 < x_2$,
\[
\Pr(x_1 < X \le x_2) = F_X(x_2) - F_X(x_1)
\]
\end{enumerate}

\subsection{Example 1: Validity of CDFs}

\textbf{Task.} Determine whether each given function is a valid CDF.

\paragraph{(a)}  
As $x \to -\infty$, $F_X(x) \to 0$.  
As $x \to \infty$, $F_X(x) \to 1$.  
The function is monotonically increasing.

\textbf{Conclusion:} Valid CDF.

\paragraph{(b)}  
The unit step function ensures
\[
F_X(x) =
\begin{cases}
0, & x < 0 \\
1 - e^{-x}, & x \ge 0
\end{cases}
\]

All bounds, monotonicity, and limits are satisfied.

\textbf{Conclusion:} Valid CDF.

\paragraph{(c)}
\[
F_X(x) = \frac{1}{\pi}\tan^{-1}(x) + \frac{1}{2}
\]

Since $\tan^{-1}(x) \in \left(-\frac{\pi}{2}, \frac{\pi}{2}\right)$, we have $F_X(x) \in (0,1)$.

As $x \to \pm\infty$, boundary conditions are violated as stated in the lecture.

\textbf{Conclusion:} Not a valid CDF.

\paragraph{(d)}  
The function grows unbounded for large $x$, violating boundedness.

\textbf{Conclusion:} Not a valid CDF.

\subsection{Example 2: Computing Probabilities from a Given CDF}

Given
\[
F_X(x) = (1 - e^{-x})u(x)
\]

\paragraph{(a)}
\[
\Pr(X > 5) = 1 - F_X(5) = e^{-5}
\]

\paragraph{(b)}
\[
\Pr(X < 5) = F_X(5) = 1 - e^{-5}
\]

\paragraph{(c)}
\[
\Pr(3 < X < 7) = F_X(7) - F_X(3)
\]

\paragraph{(d)}
Let $A = \{X > 5\}$ and $B = \{X < 7\}$. Then
\[
\Pr(A \mid B) = \frac{F_X(7) - F_X(5)}{F_X(7)}
\]

\section{Probability Density Function (PDF)}

\subsection{Definition via Limit}

For a continuous random variable $X$, the PDF at $x$ is defined as
\[
f_X(x) = \lim_{\varepsilon \to 0} \frac{\Pr(x \le X < x+\varepsilon)}{\varepsilon}
\]

\subsection{Relationship Between PDF and CDF}

\[
\Pr(x \le X < x+\varepsilon) = F_X(x+\varepsilon) - F_X(x)
\]

\[
f_X(x) = \lim_{\varepsilon \to 0} \frac{F_X(x+\varepsilon) - F_X(x)}{\varepsilon}
= \frac{dF_X(x)}{dx}
\]

\subsection{Fundamental Relationship}

The PDF of a random variable is the derivative of its CDF.  
The CDF is the integral of its PDF.

\section{Expectation of Random Variables}

The lecture outline lists this section, but no definitions, derivations, or examples are present in the provided slides.

\end{document}