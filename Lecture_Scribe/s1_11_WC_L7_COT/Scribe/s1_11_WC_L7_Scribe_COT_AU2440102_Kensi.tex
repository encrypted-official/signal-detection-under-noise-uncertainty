\documentclass{article}

\usepackage[a4paper,margin=1in]{geometry}
\usepackage{amsmath,amssymb}
\usepackage{booktabs}

\title{\textbf{CSE400 – Fundamentals of Probability in Computing}}
\author{Kensi Patel - AU2440102}
\date{7th February 2026.}

\begin{document}

\maketitle

\textbf{Lecture 7: Expectation, CDFs, PDFs, and Problem Solving}

Instructor: Dhaval Patel, PhD 

Date: January 27, 2025

\section*{Lecture Flow and Logical Dependencies}

\begin{enumerate}
    \item Random Variables (assumed known from previous lectures)
    \item Cumulative Distribution Function (CDF)
    \item Probability Density Function (PDF)
    \item Expectation of Random Variables
    \item Expectation of a function of a RV
    \item Linearity of Expectation
    \item Moments and Central Moments 
    \begin{itemize}
        \item Variance
        \item Skewness
        \item Kurtosis
    \end{itemize}
\end{enumerate}
 
\section*{1. Cumulative Distribution Function (CDF)}

\textbf{Definition}

For a random variable $X$,
\[F_X(x) = P(X \le x)\]

This definition applies to both discrete and continuous random variables.

\textbf{Properties of CDF}

For any random variable $X$:
\begin{enumerate}
    \item $0 \le F_X(x) \le 1$
    \item $F_X(x)$ is non-decreasing
    \item $\lim_{x \to -\infty} F_X(x) = 0$
    \item $\lim_{x \to +\infty} F_X(x) = 1$
    \item Right-continuous:
    \[
    F_X(x) = \lim_{h \to 0^+} F_X(x+h)
    \]
\end{enumerate}

\textbf{Probability Using CDF}

For any $a<b$:
\[P(a<X\le b) = F_X(b) - F_X(a)\]

\section*{2. Probability Density Function (PDF)}

\textbf{Definition}

For a continuous random variable $X$, the PDF $f_X(x)$ is defined as:
\[f_X(x) = \frac{d}{dx}F_X(x)\]

\textbf{Relationship Between PDF and CDF}
\[F_X(x)=\int_{\infty}^{x}f_X(t)\,d\]

\textbf{Properties of PDF}
\begin{enumerate}
    \item $f_X(x) \ge 0$ for all $x$
    \item Total area under the curve equals 1:
    \[\int_{-\infty}^{\infty} f_X(x)\,dx = 1\]
    \item Probability over an interval:
    \[P(a \le X \le b) = \int_a^b f_X(x)\,dx\]
    \item For continuous RVs:
    \[P(X=x)=0\]
\end{enumerate}

\section*{3. Expectation of Random Variables}

\textbf{Definition (Discrete RV)}

If $X$ takes values $x_i$ with probabilities $p(x_i)$:
\[E[X] = \sum_i x_i\,p(x_i)\]

\textbf{Definition (Continuous RV)}
\[E[X] = \int_{-\infty}^{\infty} x f_X(x)\,dx\]

\textbf{Interpretation}

Expectation represents the mean (average) value of the random variable.

\section*{4. Expectation of a Function of a Random Variable}

Let $Y=g(X)$

\textbf{Discrete Case}
\[E[g(X)] = \sum_i g(x_i)\,p(x_i)\]

\textbf{Continuous Case}
\[E[g(X)] = \int_{-\infty}^{\infty} g(x)f_X(x)\,dx\]

Important: No need to find the PDF of $Y$ explicitly.

\section*{5. Linearity of Expectation}

For random variables $X,Y$ and constants $a,b$:
\[E[aX+bY] = aE[X] + bE[Y]\]

\textbf{Key Exam Note}
\begin{itemize}
    \item Linearity holds regardless of independence
    \item Applies to any number of RVs
\end{itemize}

\section*{6. Moments of a Random Variable}

\textbf{n-th Moment (About Origin)}
\[E[X^n]\]

\textbf{Central Moments}

Defined about the mean $\mu = E[X]$:
\[E[(X-\mu)^n]\]

\section*{7. Variance}

\textbf{Definition}
\[\text{Var}(X) = E[(X-\mu)^2]\]

\textbf{Computational Formula}
\[\text{Var}(X) = E[X^2] - (E[X])^2\]

\textbf{Standard Deviation}
\[\sigma = \sqrt{\text{Var}(X)}\]

\section*{8. Skewness}

\textbf{Definition}

Third central moment (normalized):
\[\text{Skewness} = E\left[\left(\frac{X-\mu}{\sigma}\right)^3\right]
\]

Measures asymmetry of distribution.

\section*{9. Kurtosis}

\textbf{Definition}

Fourth central moment (normalized):
\[\text{Kurtosis} = E\left[\left(\frac{X-\mu}{\sigma}\right)^4\right]\]

Measures peakedness / tail heaviness.

\section*{Exam-Focused Summary Table}

\begin{center}
\begin{tabular}{ll}
\toprule
Concept & Formula \\
\midrule
CDF & $F_X(x)=P(X\le x)$ \\
PDF & $f_X(x)=\dfrac{d}{dx}F_X(x)$ \\
Expectation & $E[X]=\int x f_X(x)\,dx$ \\
Linearity & $E[aX+bY]=aE[X]+bE[Y]$ \\
Variance & $E[X^2]-(E[X])^2$ \\
Skewness & $E[((X-\mu)/\sigma)^3]$ \\
Kurtosis & $E[((X-\mu)/\sigma)^4]$ \\
\bottomrule
\end{tabular}
\end{center}

\end{document}
