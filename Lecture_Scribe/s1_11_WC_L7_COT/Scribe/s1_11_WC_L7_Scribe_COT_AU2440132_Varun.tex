\documentclass[12pt]{article}
\usepackage{amsmath,amssymb,geometry,graphicx}
\geometry{margin=1in}
\title{CSE400: Fundamentals of Probability in Computing\\
Lecture 7 Scribe: Expectation, CDFs, PDFs and Problem Solving}
\author{Based strictly on Lecture Slides}
\date{}

\begin{document}
\maketitle

\section{Introduction}
This lecture covers:
\begin{itemize}
\item Cumulative Distribution Function (CDF)
\item Probability Density Function (PDF)
\item Relationship between PDF and CDF
\item Expectation of Random Variables (intro outline)
\end{itemize}

The lecture develops definitions, properties, and examples of CDF and PDF, and builds intuition using analogies.

\section{Intuition: Water Tank Analogy}
The cumulative distribution function can be understood using a water tank analogy.

\subsection*{Interpretation}
\begin{itemize}
\item Height $h$ of water $\rightarrow$ value of random variable
\item Volume of water up to height $h$ $\rightarrow$ cumulative probability
\end{itemize}

Volume up to height $h$:
\[
V(h)=\int_0^h \pi R^2\,dh = (\pi R^2)h
\]

\begin{itemize}
\item $\pi R^2$ behaves like a constant density (analogous to PDF of uniform distribution)
\item Total tank volume $\pi R^2 H$ corresponds to total probability $=1$
\end{itemize}

Thus:
\begin{itemize}
\item Cumulative volume $\rightarrow$ CDF
\item Density of water per unit height $\rightarrow$ PDF
\end{itemize}

\section{Cumulative Distribution Function (CDF)}

\subsection{Definition}
The CDF of a random variable $X$ is defined as:
\[
F_X(x)=\Pr(X\le x), \quad -\infty<x<\infty
\]

Most information about the random experiment described by $X$ is determined by the behavior of $F_X(x)$.

\subsection{Properties of CDF}
\begin{enumerate}
\item \textbf{Bounds:}
\[
0 \le F_X(x) \le 1
\]

\item \textbf{Limits:}
\[
F_X(-\infty)=0,\quad F_X(\infty)=1
\]

\item \textbf{Monotonicity:}  
For $x_1<x_2$:
\[
F_X(x_1)\le F_X(x_2)
\]

\item \textbf{Interval Probability:}
\[
\Pr(x_1<X\le x_2)=F_X(x_2)-F_X(x_1)
\]
\end{enumerate}

\section{Valid CDF Identification}

\subsection*{Example 1: Determine Valid CDF}
Check validity of:

\begin{enumerate}
\item 
\[
F_X(x)=\frac{1}{2}+\frac{1}{\pi}\tan^{-1}(x)
\]
Valid CDF.

\item 
\[
F_X(x)=[1-e^{-x}]u(x)
\]
Valid CDF.

\item 
\[
F_X(x)=e^{-x^2}
\]
Invalid (does not satisfy boundary conditions).

\item 
\[
F_X(x)=x^2u(x)
\]
Invalid (does not approach 1 as $x\to\infty$).
\end{enumerate}

\section{Example Using CDF}
Given:
\[
F_X(x)=(1-e^{-x})u(x)
\]

\subsection{Find Probabilities}

\subsubsection*{1. $\Pr(X>5)$}
\[
\Pr(X>5)=1-\Pr(X\le5)
\]
\[
=1-F_X(5)
\]
\[
=1-(1-e^{-5})=e^{-5}
\]

\subsubsection*{2. $\Pr(X<5)$}
\[
\Pr(X<5)=F_X(5)=1-e^{-5}
\]

\subsubsection*{3. $\Pr(3<X<7)$}
\[
\Pr(3<X<7)=F_X(7)-F_X(3)
\]
\[
=(1-e^{-7})-(1-e^{-3})
\]
\[
=e^{-3}-e^{-7}
\]

\subsubsection*{4. Conditional Probability}
\[
\Pr(X>5\mid X<7)=\frac{\Pr(5<X<7)}{\Pr(X<7)}
\]
\[
=\frac{F_X(7)-F_X(5)}{F_X(7)}
\]

\section{Probability Density Function (PDF)}

\subsection{Definition}
For continuous random variables:
\[
f_X(x)=\lim_{\epsilon\to0}\frac{\Pr(x\le X<x+\epsilon)}{\epsilon}
\]

Using CDF:
\[
\Pr(x\le X<x+\epsilon)=F_X(x+\epsilon)-F_X(x)
\]

Thus:
\[
f_X(x)=\lim_{\epsilon\to0}\frac{F_X(x+\epsilon)-F_X(x)}{\epsilon}
\]

\subsection{Derivative Relationship}
\[
f_X(x)=\frac{dF_X(x)}{dx}
\]

\subsection{Key Result}
\begin{itemize}
\item PDF is derivative of CDF
\item CDF is integral of PDF
\end{itemize}

\[
F_X(x)=\int_{-\infty}^{x} f_X(t)\,dt
\]

\section{Conceptual Summary}
\begin{itemize}
\item CDF gives cumulative probability up to $x$
\item PDF gives density at point $x$
\item CDF must be monotonic and bounded
\item PDF integrates to 1 over entire range
\item Probability over interval found using CDF difference
\end{itemize}

\section{Expectation of Random Variables}
(Only outline provided in lecture)
\begin{itemize}
\item Definition of expectation
\item Expectation of function of random variable
\item Linearity of expectation
\item Moments and central moments
\item Variance, skewness, kurtosis
\end{itemize}

(Detailed derivations not included in provided lecture slides.)

\section{Exam Preparation Notes}
\subsection*{Must Remember}
\begin{itemize}
\item $F_X(x)=\Pr(X\le x)$
\item $0\le F_X(x)\le1$
\item $F_X(-\infty)=0,\;F_X(\infty)=1$
\item $\Pr(a<X\le b)=F_X(b)-F_X(a)$
\item $f_X(x)=\dfrac{dF_X(x)}{dx}$
\item $F_X(x)=\int_{-\infty}^{x} f_X(t)dt$
\end{itemize}

\subsection*{Typical Exam Questions}
\begin{itemize}
\item Check if function is valid CDF
\item Compute probabilities from CDF
\item Find PDF from CDF
\item Derive relation between PDF and CDF
\item Conditional probability using CDF
\end{itemize}

\end{document}
