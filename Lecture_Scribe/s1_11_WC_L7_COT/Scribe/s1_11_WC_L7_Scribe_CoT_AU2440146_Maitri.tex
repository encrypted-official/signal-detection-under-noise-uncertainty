\documentclass[11pt]{article}

% ===================== PACKAGES =====================
\usepackage[a4paper,margin=1in]{geometry}
\usepackage{amsmath,amssymb}
\usepackage{enumitem}
\usepackage{fancyhdr}
\usepackage{xcolor} 

% ===================== HEADER & FOOTER =====================
\pagestyle{fancy}
\fancyhf{}
\lhead{CSE 400: Fundamentals of Probability in Computing}
\rhead{Lecture 7 Scribe}
\cfoot{\thepage}

\title{\textbf{CSE400 -- Fundamentals of Probability in Computing}\\
\large Lecture 7: Expectation, CDFs, PDFs and Problem Solving}
\author{\textbf{Name:} Maitri Lad   \textbf{Enrollnment no.:} AU2440146}
\date{January 27, 2026}

\begin{document}

\maketitle

\section{Outline of the Lecture}

The lecture covers the following topics in order:
\begin{itemize}
    \item The Cumulative Distribution Function (CDF)
    \begin{itemize}
        \item Definition
        \item Properties
        \item Examples
    \end{itemize}
    \item The Probability Density Function (PDF)
    \begin{itemize}
        \item Definition
        \item PDF--CDF relationship
    \end{itemize}
    \item Expectation of random variables
    \begin{itemize}
        \item Definition and example
        \item Expectation of a function of a random variable
        \item Linear operations with expectation
    \end{itemize}
    \item $n^{\text{th}}$ moments and central moments of random variables
    \begin{itemize}
        \item Variance
        \item Skewness
        \item Kurtosis
    \end{itemize}
\end{itemize}

\section{CDF and PDF: Water Tank Analogy}

A water tank analogy is used to motivate the idea of CDF and PDF.

\begin{itemize}
    \item The height $h$ of water corresponds to the value of a random variable.
    \item The volume of water up to height $h$ corresponds to the cumulative probability.
\end{itemize}

Let
\begin{itemize}
    \item Tank radius $= R$
    \item Maximum height $= H$
\end{itemize}

The volume up to height $h$ is
\[
V(h)=\int_{0}^{h}\pi R^{2}\,dh=(\pi R^{2})h.
\]

Here, $\pi R^{2}$ is analogous to the PDF of a uniform distribution.

The total volume is
\[
V(H)=\pi R^{2}H,
\]
which is analogous to total probability $=1$.

\section{Cumulative Distribution Function (CDF)}

\subsection{Definition}

For a random variable $X$, the cumulative distribution function is defined as
\[
F_X(x)=\Pr(X\le x),\qquad -\infty<x<\infty.
\]

\subsection{Interpretation}

Most of the information about the random experiment described by $X$ is determined by the behavior of $F_X(x)$.

\subsection{Properties of the CDF}

\begin{itemize}
    \item \textbf{Bounds}
    \[
    0\le F_X(x)\le 1
    \]
    \item \textbf{Limits}
    \[
    F_X(-\infty)=0,\qquad F_X(\infty)=1
    \]
    \item \textbf{Monotonicity}

    For $x_1<x_2$,
    \[
    F_X(x_1)\le F_X(x_2)
    \]
    \item \textbf{Probability over an interval}

    For $x_1<x_2$,
    \[
    \Pr(x_1<X\le x_2)=F_X(x_2)-F_X(x_1).
    \]
\end{itemize}

\section{CDF -- Examples}

\subsection{Example 1: Validity of Given Functions}

Determine whether the following are valid CDFs.

\[
F_X(x)=\frac{1}{2}+\frac{1}{\pi}\tan^{-1}(x) \qquad \text{(Valid CDF)}
\]

\[
F_X(x)=[1-e^{-x}]\,u(x) \qquad \text{(Valid CDF)}
\]

\[
F_X(x)=e^{-x^{2}} \qquad \text{(Not a valid CDF)}
\]

\[
F_X(x)=x^{2}u(x) \qquad \text{(Not a valid CDF)}
\]

Here, $u(x)$ denotes the unit step function.

\subsection{Example 2: Probability Computation Using a Given CDF}

Given
\[
F_X(x)=(1-e^{-x})u(x),
\]

\begin{align*}
\Pr(X>5)
&=1-\Pr(X\le5)\\
&=1-F_X(5)=e^{-5}.
\end{align*}

\[
\Pr(X<5)=F_X(5).
\]

\[
\Pr(3<X<7)=F_X(7)-F_X(3).
\]

\subsection{Conditional Probability}

\[
\Pr(X>5\,|\,X<7)
=\frac{\Pr(5<X<7)}{\Pr(X<7)}
=\frac{F_X(7)-F_X(5)}{F_X(7)}.
\]

\section{Probability Density Function (PDF)}

\subsection{Definition (via Limiting Argument)}

For a continuous random variable $X$, the PDF is defined as
\[
f_X(x)=\lim_{\varepsilon\to 0}
\frac{\Pr(x\le X<x+\varepsilon)}{\varepsilon}.
\]

\subsection{Relationship Between PDF and CDF}

\[
\Pr(x\le X<x+\varepsilon)
=F_X(x+\varepsilon)-F_X(x).
\]

Substituting,
\[
f_X(x)=\lim_{\varepsilon\to 0}
\frac{F_X(x+\varepsilon)-F_X(x)}{\varepsilon}.
\]

Hence,
\[
f_X(x)=\frac{d}{dx}F_X(x).
\]

\subsection{Fundamental Result}

The PDF of a random variable is the derivative of its CDF.

Conversely, the CDF can be obtained by integrating the PDF.

\section{Expectation of Random Variables}

\subsection{Definition}

The expectation of a random variable represents its mean value.

Expectation is also interpreted as the first moment of a random variable.

\subsection{Expectation of a Function of a Random Variable}

For a function $g(X)$, the expectation is written as
\[
E[g(X)].
\]

\subsection{Linear Operations with Expectation}

Expectation is linear. For constants $a,b$ and random variables $X,Y$,
\[
E[aX+bY]=aE[X]+bE[Y].
\]

\section{Moments and Central Moments}

\subsection{$n^{\text{th}}$ Moment}

The $n^{\text{th}}$ moment of a random variable $X$ is
\[
E[X^{n}].
\]

\subsection{Central Moments}

Central moments are taken about the mean $\mu$,
\[
E[(X-\mu)^{n}].
\]

\subsection{Important Central Moments}

\begin{itemize}
    \item Variance: second central moment
    \item Skewness: third central moment
    \item Kurtosis: fourth central moment
\end{itemize}

\section{Logical Flow Summary for Exam Revision}

\begin{itemize}
    \item Start with the definition of the CDF
    \item Use CDF properties to test validity
    \item Compute probabilities using CDF differences
    \item Derive the PDF from the CDF
    \item Interpret expectation as a moment
    \item Apply linearity of expectation
    \item Extend to higher-order and central moments
\end{itemize}

\end{document}
