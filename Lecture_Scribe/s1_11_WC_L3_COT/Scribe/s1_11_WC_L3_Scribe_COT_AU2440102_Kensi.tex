\documentclass{article}

\usepackage[a4paper,margin=1in]{geometry}

\title{\textbf{CSE400 – Fundamentals of Probability in Computing}}

\author{\textbf{Kensi Patel - AU2440102}}

\date{\textbf{7th February 2026}}

\begin{document}

\maketitle

\textbf{Lecture 3: Introduction to Probability Theory}

\textbf{(Part-0: Course \& Learning Context)}

Instructor: Dr. Dhaval Patel 

Date: January 13, 2026


\section*{1. Administrative Context of the Lecture}

\subsection*{1.1 Course Identification}

\begin{itemize}
    \item Course Code: CSE400
    \item Course Title: Fundamentals of Probability in Computing
    \item Lecture Number: 3
    \item Lecture Theme: Introduction to Probability Theory
\end{itemize}

\textbf{Note:} This lecture primarily sets up course structure, motivation, and expectations, not formal probability axioms yet.

\section*{2. Instructor and Teaching Team}

\subsection*{2.1 Instructor}

\begin{itemize}
    \item Name: Dr. Dhaval Patel
    \item Role: Course Instructor
    \item Office: Faculty Office, Room-210
    \item Email: dhaval.patel@ahduni.edu.in
\end{itemize}

\textbf{Research Areas Mentioned}
\begin{itemize}
    \item xG Networks
    \item Applied ML/DL/RL
    \item AutoML
    \item Intelligent Transportation Systems
    \item Behaviour Modelling using AI
\end{itemize}

\subsection*{2.2 Teaching Assistants (TAs)}

\begin{itemize}
    \item Multiple B.Tech (CSE) students (3rd and 4th year)
    \item Areas include:
    \begin{itemize}
        \item Reinforcement Learning
        \item Spectrum Sensing
        \item Antenna Systems
        \item Transportation Systems
    \end{itemize}
    \item \textbf{TA hours:} To be finalized (announcement to follow)
\end{itemize}

\section*{3. Learning Philosophy Emphasized in Class}

\subsection*{3.1 Growth Mindset (Explicitly Shown)}

Statements highlighted in slides:
\begin{itemize}
    \item Failure is an opportunity to grow
    \item Challenges help learning
    \item Feedback is constructive
    \item Effort determines ability
\end{itemize}

\subsection*{3.2 Fixed Mindset (Contrasted)}

\begin{itemize}
    \item Failure defines limits
    \item Avoids challenges
    \item Belief in unchangeable ability
\end{itemize}

\textbf{Purpose in lecture:} 

To set student attitude expectations for a mathematically rigorous course.

\section*{4. Why Learn CSE400? (Motivation)}

\subsection*{4.1 Daily-Life Motivation}

Probability appears in:
\begin{itemize}
    \item Everyday conversations
    \item Decision-making under uncertainty
\end{itemize}

\subsection*{4.2 Engineering Applications (Explicit List)}

\begin{itemize}
    \item Speech Recognition
    \item Radar Systems
    \item Communication Networks
\end{itemize}

These examples justify why probability theory is foundational for computing systems.

\section*{5. Course Infrastructure}

\subsection*{5.1 Active Learning Platform}

\begin{itemize}
    \item Platform: Campuswire
    \item Usage:
    \begin{itemize}       
        \item Anonymous participation
        \item Back-channel questions during lectures
        \item Real-time polling
        \item Direct messaging (DM) with instructor/TAs
        \end{itemize}
\end{itemize}

\subsection*{5.2 Course Website}

Section-1 and Section-2 links are provided through Campuswire.

\section*{6. Lecture Schedule}

\subsection*{6.1 Lecture Sessions}

\textbf{Section-1}
\begin{itemize}
    \item 9:30 AM – 11:00 AM
    \item Tuesday and Thursday
    \item GICT Room-13
\end{itemize}

\textbf{Section-2}
\begin{itemize}
    \item 1:00 PM – 2:30 PM
    \item Tuesday and Thursday
    \item GICT Room-137
\end{itemize}

\section*{7. Communication and Support Policy}

\subsection*{7.1 Instructor Interaction}

\begin{itemize}
    \item Contact hours: 24×7 through Campuswire
    \item Preferred method: Post queries on Campuswire
    \item Private issues: Direct Message (DM)
    \item One-to-one discussion possible (student–instructor only)
\end{itemize}

\section*{8. Assessment Structure}

\subsection*{8.1 Project Component}

\begin{itemize}
    \item Weightage: 30\%
    \item Includes:
    \begin{itemize}
        \item Team formation
        \item Mathematical modeling
        \item Coding
        \item Randomized algorithms
        \item Final analysis and bounds
    \end{itemize}

\end{itemize}

\subsection*{8.2 Project Milestones}

\begin{itemize}
    \item M1: Problem formulation
    \item M2: Mathematical modeling (Random Variables, PMF/PDF, CDF, multivariate RVs, joint distributions, etc.)
    \item M3: Simulation and computation
    \item M4: Inference and randomized algorithms
    \item M5: Comparison with deterministic approaches
    \item M6: Bounds, analysis, and final submission
\end{itemize}

\section*{9. Lecture Scribe Requirement (Meta-Instruction)}

\subsection*{9.1 Definition (As per course)}

A lecture scribe must:
\begin{itemize}
    \item Reflect exact lecture content
    \item Follow the same structure and terminology
    \item Avoid:
    \begin{itemize}
        \item Simplified tutorials
        \item Extra intuition
        \item New examples
        \item Solution manuals
    \end{itemize}
\end{itemize}

\subsection*{9.2 Types of Scribes}

\begin{itemize}
    \item Lecture Scribe
    \item Project Scribe
\end{itemize}

This current document follows the Lecture Scribe format.

\section*{10. Scope Boundary of This Lecture}

\begin{itemize}
    \item No probability axioms introduced
    \item No random variables defined
    \item No PMF, PDF, or CDF derivations
    \item Focus on course framing, motivation, and learning process
\end{itemize}

Formal probability theory begins in subsequent lectures.

\section*{11. End of Lecture Notes}

\begin{itemize}
    \item Open Q\&A session
    \item No mathematical derivations in this lecture
    \item Sets foundation for upcoming probability theory lectures
\end{itemize}

\end{document}
