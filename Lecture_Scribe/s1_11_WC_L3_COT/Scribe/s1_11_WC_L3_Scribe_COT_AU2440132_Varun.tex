\documentclass[12pt]{article}

\usepackage[a4paper,margin=1in]{geometry}
\usepackage{amsmath,amsfonts,amssymb}
\usepackage{graphicx}
\usepackage[hidelinks]{hyperref}
\usepackage{enumitem}
\usepackage{titlesec}
\usepackage{setspace}

\setstretch{1.2}

\title{\textbf{CSE400: Fundamentals of Probability in Computing}\\
Lecture 3: Introduction to Probability Theory\\
\large Lecture Scribe for Exam Preparation}
\author{}
\date{}

\begin{document}
\maketitle
\tableofcontents
\newpage

\section{Course Information}

\subsection{Instructor}
\begin{itemize}
\item \textbf{Dr. Dhaval Patel}
\item Role: Instructor
\item Office: Faculty Office (Room-210)
\item Email: dhaval.patel@ahduni.edu.in
\item Areas of Interest:
\begin{itemize}
\item xG Networks
\item Applied ML / DL / RL
\item AutoML
\item Intelligent Transportation Systems
\item Life Sciences
\item Behaviour Modelling using AI
\end{itemize}
\end{itemize}

\subsection{Teaching Assistants}
\begin{itemize}
\item Deep Patel – BTech CSE (3rd Year)
\item Prapti Patel – BTech CSE (4th Year)
\item Raj Koticha – BTech CSE (4th Year)
\item Ritu Patel – BTech CSE (4th Year)
\item Rushi Moliya – BTech CSE (4th Year)
\item Ura Modi – BTech CSE (3rd Year)
\end{itemize}

\subsection{Course Platform}
\begin{itemize}
\item Active Learning Platform: Campuswire
\item Used for:
\begin{itemize}
\item Anonymous participation
\item Posting and back-channel communication
\item Real-time feedback and polling
\item Direct messaging with instructor/TAs
\end{itemize}
\end{itemize}

\subsection{Lecture Schedule}
\begin{itemize}
\item Section 1: 9:30 AM – 11:00 AM (Tuesday, Thursday), GICT Room-136
\item Section 2: 1:00 PM – 2:30 PM (Tuesday, Thursday), GICT Room-137
\end{itemize}

\subsection{Discussion and Contact}
\begin{itemize}
\item Contact hours: 24x7 through Campuswire
\item Best practice: Post queries on Campuswire
\item Private discussions via direct message
\item External engagement and counselling via email
\end{itemize}

\section{Why Study Probability in Computing?}

\subsection{Motivation}
Probability plays a major role in:
\begin{itemize}
\item Daily life conversations
\item Decision making under uncertainty
\item Analytical reasoning
\end{itemize}

\subsection{Engineering Applications}
\begin{itemize}
\item Speech Recognition
\item Radar Systems
\item Communication Networks
\end{itemize}

These systems rely heavily on probabilistic modelling and reasoning.

\section{Learning Philosophy}

\subsection{Growth Mindset}
\begin{itemize}
\item Failure is an opportunity to grow
\item Challenges help improve abilities
\item Feedback is constructive
\item Effort and attitude determine abilities
\end{itemize}

\subsection{Fixed Mindset (Contrast)}
\begin{itemize}
\item Failure defines ability limits
\item Avoid challenges
\item Give up when frustrated
\item Stick to known approaches
\end{itemize}

Students are encouraged to adopt a growth mindset.

\section{Active Learning and Participation}

Active learning is emphasised through:
\begin{itemize}
\item Online participation
\item Question-driven discussion
\item Real-time feedback
\item Collaborative problem solving
\end{itemize}

\section{Project Component (30\%)}

\subsection{Team Formation}
Deadline: January 17, 2026 (EOD)

\subsection{Major Milestones}
\begin{enumerate}
\item \textbf{M1}: Team formation, problem identification, motivation
\item \textbf{M2}: Mathematical modelling (Random variables, PMF/PDF, CDF, joint distributions)
\item \textbf{M3}: Coding and simulation
\item \textbf{M4}: Inference and randomized algorithm implementation
\item \textbf{M5}: Apply randomized algorithm to domain problem
\item \textbf{M6}: Derive bounds, analysis, final submission
\end{enumerate}

\subsection{Deliverables}
\begin{itemize}
\item Codes
\item Reports
\item Videos
\item Decision logs and documentation
\end{itemize}

\subsection{Evaluation}
\begin{itemize}
\item Continuous milestone evaluation
\item Mid-semester assessment
\item Final viva and submission
\end{itemize}

\section{Lecture Scribe Requirements}

\subsection{Types}
\begin{itemize}
\item Lecture scribe
\item Project scribe
\end{itemize}

\subsection{Lecture Scribe}
\begin{itemize}
\item Prepared by assigned groups
\item Minimum 8–10 pages
\item Must reflect lecture content
\item Include additional examples from textbooks
\end{itemize}

\subsection{Project Scribe}
\begin{itemize}
\item Decision logs
\item Constraints and alternatives
\item Evidence-based reasoning
\item Trade-off matrices
\end{itemize}

\section{Multimodal Deliverables}
Each milestone requires:
\begin{itemize}
\item 10–15 minute explanation video
\item Coding or simulation demonstration
\item Conceptual explanation of work done
\end{itemize}

\section{UG Research Programme (UGRP)}

\subsection{Philosophy}
\begin{itemize}
\item Multidisciplinary learning
\item Research-driven education
\item Experiential learning
\item 4D Model: Discover, Design, Develop, Deliver
\end{itemize}

\subsection{T-shaped Engineer Concept}
\begin{itemize}
\item Depth in one technical discipline
\item Breadth across multiple domains
\item Collaboration ability
\end{itemize}

\section{Conclusion}
This lecture introduced:
\begin{itemize}
\item Course structure and logistics
\item Importance of probability in computing
\item Learning philosophy
\item Project structure and evaluation
\item Research orientation and UGRP
\end{itemize}

Students are expected to actively participate, adopt a growth mindset, and engage deeply with probabilistic thinking throughout the course.

\end{document}
