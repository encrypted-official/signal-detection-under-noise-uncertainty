\documentclass[12pt]{article}

\usepackage[a4paper,margin=1in]{geometry}
\usepackage{amsmath,amssymb,amsthm}
\usepackage{hyperref}
\usepackage{enumitem}

\title{\textbf{CSE400: Fundamentals of Probability in Computing}\\
\large Lecture 2 Scribe (Homework Assignment--1)\\
\small Exam Preparation Notes}
\author{Dhaval Patel, PhD (Lecture Content)\\Scribed for Revision}
\date{January 8, 2026}
\usepackage{fancyhdr}

\pagestyle{fancy}
\fancyhf{} % clear default header/footer

\fancyhead[L]{Name: Parthiv Karangiya}
\fancyhead[R]{Roll No.: AU2440016}

\renewcommand{\headrulewidth}{0.4pt} % header line
\renewcommand{\footrulewidth}{0pt}

\fancyfoot[C]{\thepage} % page number in footer center

\begin{document}

\maketitle

\tableofcontents
\newpage

\section{Lecture Overview and Outline}

This lecture (CSE400 Lecture 2) focuses on:
\begin{itemize}
    \item Applied probability (continuation from Lecture 1).
    \item Formula revision:
    \begin{itemize}
        \item Basic probability rules
        \item Factorials and counting
        \item Combinations
        \item A key combinatorial identity
    \end{itemize}
    \item Problem solving: interactive walkthrough of 10 selected questions from Homework Assignment--1.
    \item Logistics and submission details (LaTeX + Gradescope workflow).
\end{itemize}

The lecture explicitly solves Questions:
\[
1,2,3,4,5,6,7,9,10,15.
\]

\section{Formula Revision}

\subsection{Factorials and Counting}

\subsubsection{Factorial Definition}

For a positive integer $n$,
\begin{equation}
n! = n(n-1)(n-2)\cdots 1.
\end{equation}

\subsubsection{Combination Formula}

The binomial coefficient is defined as:
\begin{equation}
\binom{n}{r} = \frac{n!}{r!(n-r)!}.
\end{equation}

\subsection{Combinatorial Identity}

A combinatorial identity given in the lecture is:
\begin{equation}
\binom{n+m}{r} = \sum_{k=0}^{r} \binom{n}{k}\binom{m}{r-k}.
\end{equation}

This identity will later be proved in Question 6 using a counting argument.

\subsection{Basic Probability Formulas}

\subsubsection{Complement Rule}

For any event $A$,
\begin{equation}
P(A^c) = 1 - P(A).
\end{equation}

\subsubsection{Union Rule (Inclusion--Exclusion for Two Events)}

For events $A$ and $B$,
\begin{equation}
P(A \cup B) = P(A) + P(B) - P(A \cap B).
\end{equation}

\subsubsection{Conditional Probability}

For events $A$ and $B$ with $P(B)>0$,
\begin{equation}
P(A\mid B) = \frac{P(A\cap B)}{P(B)}.
\end{equation}

\subsubsection{Classical Probability Formula}

\begin{equation}
P(\text{Event}) = \frac{\text{Favorable outcomes}}{\text{Total outcomes}}.
\end{equation}

\subsection{Binomial Distribution}

The lecture provides the binomial tail probability:
\begin{equation}
P(X \ge r) = \sum_{k=r}^{n} \binom{n}{k} p^k (1-p)^{n-k}.
\end{equation}

\subsection{Law of Total Probability}

If $\{B_i\}$ is a partition of the sample space, then:
\begin{equation}
P(A) = \sum_i P(A\mid B_i)P(B_i).
\end{equation}

\subsection{Bayes' Theorem}

Bayes' theorem is stated as:
\begin{equation}
P(A\mid B) = \frac{P(B\mid A)P(A)}{P(B)}.
\end{equation}

\newpage

\section{Problem Solving: Homework Assignment--1 (Selected Questions)}

\subsection{Question 1}

\subsubsection{Problem Statement}

A total of $28\%$ of American males smoke cigarettes, $7\%$ smoke cigars, and $5\%$ smoke both cigars and cigarettes.
\begin{enumerate}[label=(\alph*)]
    \item What percentage of males smokes neither cigars nor cigarettes?
    \item What percentage smokes cigars but not cigarettes?
\end{enumerate}

\subsubsection{Solution}

Let:
\[
S_1 = \{\text{male smokes cigarettes}\}, \qquad S_2 = \{\text{male smokes cigars}\}.
\]

Given:
\[
P(S_1)=0.28,\qquad P(S_2)=0.07,\qquad P(S_1\cap S_2)=0.05.
\]

We are asked:
\begin{enumerate}[label=(\alph*)]
    \item $P(\text{neither})$
    \item $P(\text{cigars but not cigarettes})$
\end{enumerate}

\paragraph{Step 1: Probability of at least one (inclusion--exclusion).}

\begin{align}
P(S_1\cup S_2)
&= P(S_1) + P(S_2) - P(S_1\cap S_2) \\
&= 0.28 + 0.07 - 0.05 \\
&= 0.30.
\end{align}

\paragraph{(a) Probability of neither (complement).}

\begin{align}
P(\text{neither})
&= 1 - P(S_1\cup S_2) \\
&= 1 - 0.30 \\
&= 0.70.
\end{align}

\paragraph{(b) Probability of cigars but not cigarettes.}

\begin{align}
P(S_2\setminus S_1)
&= P(S_2) - P(S_1\cap S_2) \\
&= 0.07 - 0.05 \\
&= 0.02.
\end{align}

\subsection{Question 2}

\subsubsection{Problem Statement}

In how many ways can 8 people be seated in a row if:
\begin{enumerate}[label=(\alph*)]
    \item there are no restrictions on the seating arrangement?
    \item persons A and B must sit next to each other?
    \item there are 4 men and 4 women and no 2 men or 2 women can sit next to each other?
    \item there are 5 men and they must sit next to one another?
    \item there are 4 married couples and each couple must sit together?
\end{enumerate}

\subsubsection{Solution}

\paragraph{(a) No restriction.}

All permutations of 8 distinct people:
\begin{equation}
8! = 40320.
\end{equation}

\paragraph{(b) A and B must sit together.}

Treat $AB$ as one block, so we have 7 objects to arrange.
Within the block, A and B can swap positions.

\begin{equation}
7! \times 2! = 10080.
\end{equation}

\paragraph{(c) 4 men and 4 women alternating.}

There are two valid gender patterns:
\begin{itemize}
    \item start with man (M W M W M W M W)
    \item start with woman (W M W M W M W M)
\end{itemize}

For each pattern, arrange men in $4!$ ways and women in $4!$ ways.

\begin{equation}
2 \times (4!\times 4!) = 1152.
\end{equation}

\paragraph{(d) 5 men sit together.}

Treat the 5 men as one block. Along with the 3 women, there are $4$ objects to arrange.
Inside the men block, men can be permuted in $5!$ ways.

\begin{equation}
4! \times 5! = 2880.
\end{equation}

\paragraph{(e) 4 married couples sit together.}

Treat each couple as one unit. Then there are $4$ units to arrange: $4!$ ways.
Within each couple, there are $2$ internal arrangements.

\begin{equation}
4! \times 2^4 = 384.
\end{equation}

\subsection{Question 3}

\subsubsection{Problem Statement}

A class in probability theory consists of 6 men and 4 women. An examination is given and
the students are ranked according to their performance. Assume that no two students obtain the same score.
\begin{enumerate}[label=(\alph*)]
    \item How many different rankings are possible?
    \item If the men are ranked among themselves and the women among themselves, how many different rankings are possible?
\end{enumerate}

\subsubsection{Solution}

\paragraph{(a) Total rankings.}

There are $10$ students total, and a ranking is a permutation.

\begin{equation}
10! = 3,628,800.
\end{equation}

\paragraph{(b) Rankings within genders.}

Permute the 6 men among themselves and the 4 women among themselves.

\begin{equation}
6!\times 4! = 17,280.
\end{equation}

\noindent Note (as stated in lecture): This counts arrangements that differ only by permuting people of the same gender.

\subsection{Question 4}

\subsubsection{Problem Statement}

A committee of 5 is to be selected from a group of 6 men and 9 women. If the selection
is made randomly, what is the probability that the committee consists of 3 men and 2 women?

\subsubsection{Solution}

\paragraph{Step 1: Count total possible committees.}

Total people:
\[
6+9 = 15.
\]

Total ways to choose 5 from 15:
\begin{equation}
\binom{15}{5} = 3003.
\end{equation}

\paragraph{Step 2: Count favorable committees.}

Choose 3 men from 6 and 2 women from 9:
\begin{align}
\binom{6}{3}\binom{9}{2}
&= 20\times 36 \\
&= 720.
\end{align}

\paragraph{Step 3: Compute probability.}

\begin{align}
P
&= \frac{720}{3003} \\
&= \frac{240}{1001} \\
&\approx 0.23976.
\end{align}

\subsection{Question 5}

\subsubsection{Problem Statement}

In order to play a game of basketball, 10 children at a playground divide themselves into
two teams of 5 each. How many different divisions are possible?

\subsubsection{Solution}

We divide 10 children into two equal teams of 5.

\paragraph{Step 1: Choose 5 children for the first team.}

\begin{equation}
\binom{10}{5} = 252.
\end{equation}

\paragraph{Step 2: Correct for unlabeled teams.}

Teams are unlabeled, meaning swapping teams does not create a new division.
Hence each division is counted twice.

\begin{equation}
\frac{252}{2} = 126.
\end{equation}

Therefore, there are:
\[
126
\]
distinct unlabeled splits.

\subsection{Question 6}

\subsubsection{Problem Statement}

Prove that:
\begin{equation}
\binom{n+m}{r}
=
\binom{n}{0}\binom{m}{r}
+
\binom{n}{1}\binom{m}{r-1}
+
\cdots
+
\binom{n}{r}\binom{m}{0}.
\end{equation}

Hint: Consider selecting a group of size $r$ from $n$ men and $m$ women.

\subsubsection{Solution}

This is Vandermonde's identity:
\begin{equation}
\binom{n+m}{r} = \sum_{k=0}^{r}\binom{n}{k}\binom{m}{r-k}.
\end{equation}

\paragraph{Combinatorial proof (as in lecture).}

Suppose we want to choose $r$ items from two groups:
\begin{itemize}
    \item first group has size $n$
    \item second group has size $m$
\end{itemize}

If exactly $k$ items are chosen from the first group, then:
\[
r-k
\]
must be chosen from the second group.

\paragraph{Step 1: Count selections with exactly $k$ from first group.}

Number of ways:
\begin{equation}
\binom{n}{k}\binom{m}{r-k}.
\end{equation}

\paragraph{Step 2: Sum over all valid $k$.}

Summing for $k=0$ to $r$ counts all possible ways to choose $r$ objects from the combined $n+m$ objects:
\begin{equation}
\binom{n+m}{r} = \sum_{k=0}^{r}\binom{n}{k}\binom{m}{r-k}.
\end{equation}

Thus the identity is proved.

\subsection{Question 7}

\subsubsection{Problem Statement}

Suppose that 10 fish are caught at a lake that contains 5 distinct types of fish.
\begin{enumerate}[label=(\alph*)]
    \item How many different outcomes are possible, where an outcome specifies the numbers of caught fish of each of the 5 types?
    \item How many outcomes are possible when 3 of the 10 fish caught are trout?
    \item How many when at least 2 of the 10 are trout?
\end{enumerate}

\subsubsection{Solution}

Let:
\[
x_1+x_2+x_3+x_4+x_5 = 10,
\]
where each $x_i$ is the number of fish caught of type $i$.

\paragraph{(a) Total number of outcomes.}

This is the number of non-negative integer solutions to the equation above.

Using stars-and-bars:
\begin{align}
\text{Number of solutions}
&= \binom{10+5-1}{5-1} \\
&= \binom{14}{4} \\
&= 1001.
\end{align}

\paragraph{(b) Exactly 3 trout.}

Fix the trout variable to be 3. Then the remaining fish count is:
\[
10-3 = 7
\]
distributed among the remaining 4 types.

So:
\[
x_1+x_2+x_3+x_4 = 7.
\]

Number of non-negative solutions:
\begin{align}
\binom{7+4-1}{4-1}
&= \binom{10}{3} \\
&= 120.
\end{align}

\paragraph{(c) At least 2 trout.}

The lecture gives the computation as:
\begin{equation}
\text{At least 2} = 1001 - (286+220) = 495.
\end{equation}

Thus, the number of outcomes with at least 2 trout is:
\[
495.
\]

\subsection{Question 9}

\subsubsection{Problem Statement}

A board has 16 squares (arranged in a $4\times 4$ grid). Out of these 16 squares, two squares are chosen at random.
What is the probability that they have no side in common?

\subsubsection{Solution}

We pick a pair of distinct squares uniformly at random.

\paragraph{Step 1: Count total unordered pairs.}

\begin{equation}
\binom{16}{2} = 120.
\end{equation}

\paragraph{Step 2: Count pairs that share a side (adjacent).}

Horizontal adjacency:
\begin{itemize}
    \item Each row has 3 horizontal adjacent pairs.
    \item There are 4 rows.
\end{itemize}

So horizontal adjacent pairs:
\[
3\times 4 = 12.
\]

Vertical adjacency:
\begin{itemize}
    \item Each column has 3 vertical adjacent pairs.
    \item There are 4 columns.
\end{itemize}

So vertical adjacent pairs:
\[
3\times 4 = 12.
\]

Total adjacent pairs:
\[
12+12 = 24.
\]

\paragraph{Step 3: Compute probability of no common side.}

\begin{align}
P
&= 1 - \frac{24}{120} \\
&= 1 - \frac{1}{5} \\
&= \frac{4}{5}.
\end{align}

\subsection{Question 10}

\subsubsection{Problem Statement}

A multiple choice examination has 5 questions. Each question has three alternative answers, of which exactly one is correct.
The probability that a student will get 4 or more correct answers just by guessing is:

\subsubsection{Solution}

Let:
\[
X \sim \text{Bin}(n=5, p=\tfrac{1}{3}).
\]

We need:
\[
P(X\ge 4).
\]

\paragraph{Step 1: Expand into two cases.}

\[
P(X\ge 4) = P(X=4) + P(X=5).
\]

\paragraph{Step 2: Compute $P(X=4)$.}

\begin{equation}
P(X=4) = \binom{5}{4}\left(\frac{1}{3}\right)^4\left(\frac{2}{3}\right).
\end{equation}

\paragraph{Step 3: Compute $P(X=5)$.}

\begin{equation}
P(X=5) = \binom{5}{5}\left(\frac{1}{3}\right)^5.
\end{equation}

\paragraph{Step 4: Add the two terms.}

\begin{align}
P(X\ge 4)
&= \binom{5}{4}\left(\frac{1}{3}\right)^4\left(\frac{2}{3}\right)
+
\left(\frac{1}{3}\right)^5 \\
&= \frac{11}{243}.
\end{align}

\subsection{Question 15}

\subsubsection{Problem Statement}

A student appears for a quiz consisting of only true--false type questions. The student knows the answers of some questions and guesses the answers for the remaining questions.

Whenever the student knows the answer, he gives the correct answer.

The probability of the student giving the correct answer, given that he has guessed it, is $\frac{1}{2}$.

The probability of the answer being guessed, given that the student's answer is correct, is $\frac{1}{6}$.

Then the probability that the student knows the answer of a randomly chosen question is:

\subsubsection{Solution}

Define events:
\[
K = \{\text{student knows the answer}\}, \qquad
G = \{\text{student guesses the answer}\}, \qquad
C = \{\text{student answers correctly}\}.
\]

Given:
\[
P(C\mid K)=1, \qquad P(C\mid G)=\frac{1}{2}, \qquad P(G\mid C)=\frac{1}{6}.
\]

Let:
\[
P(K)=x \quad \Rightarrow \quad P(G)=1-x.
\]

\paragraph{Step 1: Compute $P(C)$.}

\begin{align}
P(C)
&= P(C\mid K)P(K) + P(C\mid G)P(G) \\
&= (1)(x) + \left(\frac{1}{2}\right)(1-x) \\
&= x + \frac{1}{2}(1-x) \\
&= \frac{1}{2}(1+x).
\end{align}

\paragraph{Step 2: Apply Bayes' Theorem.}

\begin{align}
P(G\mid C)
&= \frac{P(C\mid G)P(G)}{P(C)} \\
&= \frac{\left(\frac{1}{2}\right)(1-x)}{\left(\frac{1}{2}\right)(1+x)} \\
&= \frac{1-x}{1+x}.
\end{align}

But we are given:
\[
P(G\mid C)=\frac{1}{6}.
\]

Thus:
\begin{align}
\frac{1}{6}
&= \frac{1-x}{1+x}.
\end{align}

\paragraph{Step 3: Solve for $x$.}

\begin{align}
1+x &= 6(1-x) \\
1+x &= 6 - 6x \\
7x &= 5 \\
x &= \frac{5}{7}.
\end{align}

Therefore,
\[
P(K) = \frac{5}{7}.
\]

\newpage

\section{Lecture Closing: Next Steps and Submission Instructions}

The lecture concludes with the following statement:

\begin{itemize}
    \item In-class discussion finished: Questions 1, 2, 3, 4, 5, 6, 7, 9, 10, and 15 have been covered.
    \item Student task: Complete solutions for Q8, Q11, Q12, Q13, and Q14.
    \item Scribe all 15 solutions into the provided LaTeX template.
    \item Submission: Upload PDF to Gradescope by Jan 10, 11:59 PM.
\end{itemize}

\section{End of Lecture Scribe}

This document reproduces exactly the formulas, solution steps, and derivations presented in the lecture slides, in the same order and structure as shown.

\end{document}
